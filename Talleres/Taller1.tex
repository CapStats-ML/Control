\documentclass[11pt]{article}\usepackage[]{graphicx}\usepackage[table]{xcolor}
% maxwidth is the original width if it is less than linewidth
% otherwise use linewidth (to make sure the graphics do not exceed the margin)
\makeatletter
\def\maxwidth{ %
  \ifdim\Gin@nat@width>\linewidth
    \linewidth
  \else
    \Gin@nat@width
  \fi
}
\makeatother

\definecolor{fgcolor}{rgb}{0.345, 0.345, 0.345}
\newcommand{\hlnum}[1]{\textcolor[rgb]{0.686,0.059,0.569}{#1}}%
\newcommand{\hlstr}[1]{\textcolor[rgb]{0.192,0.494,0.8}{#1}}%
\newcommand{\hlcom}[1]{\textcolor[rgb]{0.678,0.584,0.686}{\textit{#1}}}%
\newcommand{\hlopt}[1]{\textcolor[rgb]{0,0,0}{#1}}%
\newcommand{\hlstd}[1]{\textcolor[rgb]{0.345,0.345,0.345}{#1}}%
\newcommand{\hlkwa}[1]{\textcolor[rgb]{0.161,0.373,0.58}{\textbf{#1}}}%
\newcommand{\hlkwb}[1]{\textcolor[rgb]{0.69,0.353,0.396}{#1}}%
\newcommand{\hlkwc}[1]{\textcolor[rgb]{0.333,0.667,0.333}{#1}}%
\newcommand{\hlkwd}[1]{\textcolor[rgb]{0.737,0.353,0.396}{\textbf{#1}}}%
\let\hlipl\hlkwb

\usepackage{framed}
\makeatletter
\newenvironment{kframe}{%
 \def\at@end@of@kframe{}%
 \ifinner\ifhmode%
  \def\at@end@of@kframe{\end{minipage}}%
  \begin{minipage}{\columnwidth}%
 \fi\fi%
 \def\FrameCommand##1{\hskip\@totalleftmargin \hskip-\fboxsep
 \colorbox{shadecolor}{##1}\hskip-\fboxsep
     % There is no \\@totalrightmargin, so:
     \hskip-\linewidth \hskip-\@totalleftmargin \hskip\columnwidth}%
 \MakeFramed {\advance\hsize-\width
   \@totalleftmargin\z@ \linewidth\hsize
   \@setminipage}}%
 {\par\unskip\endMakeFramed%
 \at@end@of@kframe}
\makeatother

\definecolor{shadecolor}{rgb}{.97, .97, .97}
\definecolor{messagecolor}{rgb}{0, 0, 0}
\definecolor{warningcolor}{rgb}{1, 0, 1}
\definecolor{errorcolor}{rgb}{1, 0, 0}
\newenvironment{knitrout}{}{} % an empty environment to be redefined in TeX

\usepackage{alltt}
\usepackage[table]{xcolor}
\usepackage[utf8]{inputenc}
\usepackage{amsfonts, amssymb, amsmath}
\usepackage{graphicx}
\usepackage{hyperref}
\usepackage{verbatim}
\usepackage{listings}
\usepackage{float}
\usepackage[a4paper, left=25mm, right=25mm, top=25mm, bottom=25mm]{geometry}
\usepackage{mathrsfs}
\usepackage{babel}

\setlength{\parindent}{0pt}


\title{UNIVERSIDAD NACIONAL DE COLOMBIA SEDE BOGOTA\\ CONTROL ESTADISTICO DE CALIDAD\\Taller1: Cartas de control }
\author{Cesar Augusto Prieto Sarmiento - CC: 1065843742\\
        Daniel Santiago Guzman Villanueva - CC: \\
        Cristian Camilo Prieto Zambrano - CC: }

\date{\today}
\IfFileExists{upquote.sty}{\usepackage{upquote}}{}
\begin{document}

\maketitle

\section*{Ejercicio 1}
Sea $X \sim N(\mu, \sigma)$ una característica de calidad. Mediante simulaciones, establezca el comportamiento del ARL (en control y fuera de él) de las Cartas $R$ y $S$ para observaciones normales con límites $3 \sigma$ y muestras de tamaño (a) $n=3$ y (b) $n=10$. ¿Qué regularidades observa?

\section*{Ejercicio 2}
Sea $X \sim N(\mu, \sigma)$ una característica de calidad. Se sabe que los valores objetivo de los parámetros del proceso son $\mu=\mu_0$ y $\sigma=\sigma_0$. Construir las curvas OC de la Carta $S^2$ con límites de probabilidad. Interpretar los resultados.

\section*{Ejercicio 4}
Sea $X \sim N(\mu_0, \sigma_0)$ una característica de calidad. Se pide:
\begin{enumerate}
    \item[a)] Mediante simulaciones, establezca el comportamiento del ARL de la Carta $\bar{X}$ con límites tres sigma para observaciones normales.
    \item[b)] Genere 20 subgrupos racionales de tamaño $n=3$ provenientes de $X$. Asúmase que el proceso es estable en cuanto a dispersión y con los subgrupos iniciales, construya la carta $\bar{X}$ como es habitual hasta verificar la estabilidad del proceso. Establezca el comportamiento del ARL para la carta que se obtiene del análisis de Fase I realizado.
    \item[c)] Repetir lo indicado en el literal (b) con 50 subgrupos racionales de tamaño $n=3$. Comente los resultados.
\end{enumerate}

\section*{Ejercicio 5}
Calcular el ARL de la Carta $\bar{X}$ mediante cadenas de Markov. Diseñar la carta con límites de control ubicados a tres desviaciones estándar de la media y dividiendo la región de control estadístico en franjas de ancho igual a una desviación estándar.


\end{document}
